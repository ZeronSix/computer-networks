% !TeX spellcheck = ru_RU
\section{Канальный уровень}
\subsection{Структура кадра}
Блоки, в которых передаются данные в Ethernet, называются кадрами или фреймами (англ. frame). Их структура изображена в таблице~\ref{tbl:ethernet_frame_structure}. Рассмотрим функции всех полей в кадре более подробно. 
\begin{table}[h!]
	\centering
	\begin{tabular}{|c|c|c|c|c|}
		\hline
		\begin{tabular}[c]{@{}c@{}}DAddr\\ 6 байт\end{tabular} & \begin{tabular}[c]{@{}c@{}}SAddr\\ 6 байт\end{tabular} & \begin{tabular}[c]{@{}c@{}}EtherType\\ 2 байта\end{tabular} & \begin{tabular}[c]{@{}c@{}}Data\\ 46-1500 байт\end{tabular} & \begin{tabular}[c]{@{}c@{}}FCS\\ 4 байта\end{tabular} \\ \hline
	\end{tabular}
	\caption{Структура кадра Ethernet}
	\label{tbl:ethernet_frame_structure}
\end{table}
\begin{itemize}
	\item $DAddr$ и $SAddr$ --- это адреса получателя и отправителя. Каждое устройство получает при выпуске свой уникальный MAC-адрес, состоящий из 6 байт (см. рис~\ref{fig:mac_address_structure}). Первые три байта содержат уникальный идентификатор компании-производителя, которая сама назначает значения последних трёх байтов каждому своему устройству. Стоит отметить, что каждый производитель может иметь несколько уникальных идентификаторов. 
	\item Поле $EtherType$ необходимо для определения того, данные какого протокола более высокого уровня лежат внутри кадра.
	\item $Data$ содержит непосредственно сами передаваемые данные (например, пакет IP).
	\item $FCS$ --- это контрольная сумма. Она необходима для определения наличия ошибок, возникших при передаче кадра.
\end{itemize}

Также в самом начале фрейма находится преамбула, которая используется для синхронизации двух устройств между собой.

Длина кадров меняется в пределах от 64 до 1518 байт. Ограничение сверху нужно для того, чтобы устройство не занимало канал на слишком большой промежуток времени, в то время как ограничение снизу необходимо для возможности определения коллизий (т.е. попыток одновременной передачи данных двумя устройствами). Если кадр будет слишком коротким, устройство не успеет задетектировать наличие передачи от другого устройства из-за конечной скорости распространения сигналов в среде. Устройство детектирует коллизии путем сравнения передаваемого сигнала с сигналом, реально наблюдаемым в среде. При коллизии передаваемые сигналы с двух устройств наложатся друг на друга, и наблюдаемый сигнал на каждом устройстве будет отличаться от передаваемого.
\begin{figure}[t!]
	\centering
	\begin{tabular}{cc}
		\hline
		\multicolumn{1}{|c|}{3 байта} & \multicolumn{1}{c|}{3 байта} \\ \hline
		идентификатор производителя & идентификатор устройства
	\end{tabular}
	\caption{Структура MAC-адреса}
	\label{fig:mac_address_structure}
\end{figure}
\subsection{Виды MAC-адресов}
Существует три вида MAC-адресов:
\begin{itemize}
	\item unicast адреса, предназначенные для передачи одному одному устройству;
	\item broadcast адреса, используемые для широковещательной рассылки;
	\item multicast адреса, используемые для передачи нескольким устройствам.
\end{itemize}

Широковещательная рассылка означает, что фрейм предназначается всем устройствам в сегменте. Широковещательный адрес всего один -- $FF:FF:FF:FF:FF:FF$. 

В данном курсе multicast адреса особо рассматриваться не будут. Однако стоит отметить, что если младший бит первого байта MAC-адреса установлен в 1, то мы имеем дело с multicast адресом. 
\subsection{Методы доступа к каналу}
Для доступа к каналу (среде передачи данных) в Ethernet используется CSMA/CD (англ. Carrier Sense Multiple Access with Collision Detection — множественный доступ с прослушиванием несущей и обнаружением коллизий). Суть этого метода заключается в следующем.

При неудачной попытке передачи устройство выжидает случайный период времени, равномерно распределенный на интервале $[1, 2^n-1] \cdot T$, где $T$ --- время передачи 512 бит, а $n$ --- номер попытки отправки фрейма. Чтобы время ожидания начала передачи не было слишком большим, максимальное количество неудачный попыток ограничено четырьмя. 

Очевидно, что с ростом количества активных узлов в сегменте работоспособность у данной схемы сильно проседает.

\subsection{Литература}
\begin{itemize}
	\item ICND1 \cite{icnd1eng}, Глава 2;
	\item Олифер \cite{olipher}, Глава 12, с. 360-372.
\end{itemize}
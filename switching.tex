% !TeX spellcheck = ru_RU
\section{Устройства 2 уровня модели OSI}
\subsection{Широковещательные и коллизионные домены}
\textbf{Широковещательный домен} - часть сети, в которой широковещательнный фрейм, посланный с любого устройства, достигнет всех остальных устройств в данном домене.
\textbf{Коллизионный домен} - часть сети, в которой между двумя устройствами может произойти коллизия.
\subsection{Мосты и мостовые таблицы}
Мосты, также как и повторители, предназначены для связи двух сегментов Ethernet между собой. Однако, логика работы этих устройств сильно отличается.

\begin{table}[h!]
	\centering
	\begin{tabular}{|c|c|c|}
		\hline
		Порт & MAC & Время \\ \hline
		1 & $MAC_a$ & $t_1$ \\ \hline
		1 & $MAC_b$ & $t_2$ \\ \hline
		1 & $MAC_c$ & $t_3$ \\ \hline
		2 & $MAC_d$ & $t_4$ \\ \hline
		\multicolumn{3}{|c|}{...} \\ \hline
	\end{tabular}
	\caption{Пример мостовой таблицы}
	\label{tbl:cam_table}
\end{table}
Каждый мост строит MAC/CAM/мостовую таблицу. Эта таблица предназначена для оптимизации передачи фрейма. Ее структура показана в табл.~\ref{tbl:cam_table}. В поле $Port$ содержится номер порта, с которого был получен фрейм с MAC-адресом, указанным в поле $MAC$. Таблица может заполняться как автоматически самим мостом, так и администратором. На размер мостовой таблицы имеются ограничения: от 8000 записей в простых устройствах до 128000 в более дорогих.

\begin{figure}[h!]
	\centering
	\begin{tikzpicture}
		\draw (-4.5, 0) -- (0,0);
		\draw (0, 0) -- (4.5,0);
		
		\draw (-4, 0) -- (-4, 1);
		\node at (-4, 1) [label=above:A] {\includegraphics{pic/pc.eps}};
		
		\draw (-3, -1) -- (-3, 0);
		\node at (-3, -1) [label=below:B] {\includegraphics{pic/pc.eps}};
		
		\draw (-2, 0) -- (-2, 1);
		\node at (-2, 1) [label=above:C] {\includegraphics{pic/pc.eps}};
		
		\draw (4, 0) -- (4, 1);
		\node at (4, 1) [label=above:F] {\includegraphics{pic/pc.eps}};
		
		\draw (3, -1) -- (3, 0);
		\node at (3, -1) [label=below:E] {\includegraphics{pic/pc.eps}};
		
		\draw (2, 0) -- (2, 1);
		\node at (2, 1) [label=above:D] {\includegraphics{pic/pc.eps}};
		
		\draw[->] (-4.9, 1.5) -- (-4.9, 0.45);
		\draw[<-] (-3.9, -1.5) -- (-3.9, -0.45);
		\draw[<-] (-2.9, 1.5) -- (-2.9, 0.45);
		\draw[->] (-1.9, -0.3) -- (-0.8, -0.3);
		
		\node at (0, 0) [label=above:Мост] {\includegraphics{pic/bridge.eps}};
		\node at (-0.6, 0.1) [above left] {1};
		\node at (0.7, 0.1) [above right] {2};
	\end{tikzpicture}
	\caption{Пример применения моста}
	\label{fig:bridge}
\end{figure}
Рассмотрим принцип работы моста на примере (см. рис.~\ref{fig:bridge}). 
\begin{itemize}
	\item Пусть узел A посылает фрейм узлу B. В этом случае мост видит, что A и B в одном сегменте, и игнорирует данный фрейм, так как узел B уже получил его. 
	\item Если узел А пошлёт фрейм узлу D, то мост передаст фрейм в тот сегмент, где находится узел D. Если A и D начнут передачу данных одновременно, то коллизии не произойдет. При коллизиях мост ведет себя так же, как и обычный узел с удержанием фрейма в буфере неопределённое время.
	\item Если узел C переедет в тот сегмент, где находится D, то через некоторое время мост пометит его запись в таблице как устаревшую и удалит ее из таблицы. Аналогичная ситуация произойдёт, если узел просто выключится.
	\item Широковещательные фреймы передаются мостом также и в другой сегмент сети.
	\item Если на мост придёт фрейм для неизвестного получателя, то он обработается как широковещательный.
\end{itemize}

Когда таблица заполнена и в сети появляется новое устройство, то поведение моста будет зависеть от архитектуры конкретного устройства: либо будут удаляться старые записи (даже если время не истекло), либо новые записи не будут добавлены.
\subsection{Коммутаторы}
Коммутатор (свитч) --- это по своей сути многопортовый мост. Именно эти устройства составляет основу современных локальный сетей Ethernet. 
\begin{figure}[h!]
	\centering
	\begin{tikzpicture}
		\draw (-4, 0) -- (0,0);
		\draw (0, 0) -- (4,0);
		\draw (0, 0) -- (0, -3);
		
		\draw (-3, 0) -- (-3, 1);
		\node at (-3, 1) [label=above:A] {\includegraphics{pic/pc.eps}};
		
		\draw (-2.25, -1) -- (-2.25, 0);
		\node at (-2.25, -1) [label=below:B] {\includegraphics{pic/pc.eps}};
		
		\draw (3, 0) -- (3, 1);
		\node at (3, 1) [label=above:E] {\includegraphics{pic/pc.eps}};
		
		\draw (2, -1) -- (2, 0);
		\node at (2, -1) [label=below:D] {\includegraphics{pic/pc.eps}};
		
		\draw (0, -2) -- (-1, -2);
		\node at (-1, -2) [label=below:C] {\includegraphics{pic/pc.eps}};
		
		\node at (0, 0) [label=above:Коммутатор] {\includegraphics{pic/switch.eps}};
	\end{tikzpicture}
	\caption{Коммутатор}
	\label{fig:switch}
\end{figure}

С внедрением коммутаторов проблемы в сетях стали возникать только в пределах одного сегмента, поэтому количество устройств, подключенных в одном сегменте, стали уменьшать. Если к каждому сегменту подключить лишь один узел и использовать полный дуплекс, то коллизий в такой сети попросту \textbf{не будет}.

Также, как и в мостах, в коммутаторах есть буферы, в которых еще не отправленные фреймы хранятся некоторое время. Благодаря этому коммутаторы, в отличие от концентратора, позволяют подключать разноскоростное оборудование.

\begin{figure}[h!]
	\centering
	\begin{minipage}[h]{0.45\linewidth}
		\center {
			\begin{tabular}{|c|c|c|}
				\hline
				Порт & MAC & Время \\ \hline
				1 & $A$ & $t_1$ \\ \hline
				2 & $B$ & $t_2$ \\ \hline
				3 & $C$ & $t_3$ \\ \hline
			\end{tabular}
		}
	\end{minipage}
	\hfill
	\begin{minipage}[h]{0.45\linewidth}
		\center {
			\begin{tikzpicture}
					\draw (-2, -2) -- (0,0);
					\draw (0, -2) -- (0,0);
					\draw (2, -2) -- (0,0);
			
					\node at (-2, -2) [label=below:A] {\includegraphics{pic/pc.eps}};
					\node at (0, -2) [label=below:B] {\includegraphics{pic/pc.eps}};
					\node at (2, -2) [label=below:C] {\includegraphics{pic/pc.eps}};
			
					\node at (0, 0) {\includegraphics{pic/switch.eps}};
					\node at (-0.65, -0.35) [below left] {1};
					\node at (0, -0.4) [below left] {2};
					\node at (0.65, -0.35) [below right] {3};
			\end{tikzpicture}
		}
	\end{minipage}
	\hfill
	\vfill
	\caption{Пример, иллюстрирующий проблемы безопасности}
	\label{fig:switch_security}
\end{figure}
Использование коммутаторов также порождает некоторые проблемы, связанные с безопасностью. Рассмотрим следующий пример (см. рис.~\ref{fig:switch_security}). Если узел C подменит свой MAC-адрес с C на B, то фреймы, предназначенные узлу B, будут приходить узлу C до тех пор, пока узел B не обновит запись в мостовой таблице своим фреймом с правильным MAC-адресом.
\section{Режимы коммутации сетей}
Существует три режима коммутации сетей:
\begin{itemize}
	\item \textbf{store and forward} (фрейм копируется в буфер целиком, и только после этого начинается обработка: проверка контрольной суммы, а затем уже работа с мостовой таблицей);
	\item \textbf{cut-through} (коммутатор получает первые 6 байт от начального заголовка фрейма (т.е. MAC-адрес получателя) и сразу проверяет мостовую таблицу и отправляет фрейм);
	\item \textbf{fragment-free} (коммутатор принимает первые 64 байт и затем работает как cut-through, однако, в отличие от последнего, этот режим спасает от фреймов, поврежденных во время коллизий).
\end{itemize}

\section{Литература}
\begin{itemize}
	\item ICND1 \cite{icnd1eng}, Глава 6;
	\item Олифер \cite{olipher}, Глава 13.
\end{itemize}
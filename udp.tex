% !TeX spellcheck = ru_RU
\section{Протокол UDP}
\subsection{Заголовок}
Структура заголовка UDP представлена в таблице~\ref{tbl:udp_header}.
\begin{table}[h!]
	\centering
	\begin{tabular}{|c|c|l|c|l|}
		\hline
		Бит & \multicolumn{2}{c|}{0-15} & \multicolumn{2}{c|}{16-31} \\ \hline
		0 & \multicolumn{2}{c|}{\begin{tabular}[c]{@{}c@{}}Source Port\\ порт отправителя\end{tabular}} & \multicolumn{2}{c|}{\begin{tabular}[c]{@{}c@{}}Destination Port\\ порт получателя\end{tabular}} \\ \hline
		32 & \multicolumn{2}{c|}{\begin{tabular}[c]{@{}c@{}}Length\\ длина датаграмм\end{tabular}} & \multicolumn{2}{c|}{\begin{tabular}[c]{@{}c@{}}Checksum\\ Контрольная сумма\end{tabular}} \\ \hline
		64 & \multicolumn{4}{c|}{Данные} \\ \hline
	\end{tabular}
	\caption{Структура заголовка UDP}
	\label{tbl:udp_header}
\end{table}
\subsection{Описание}
UDP - более простой по сравнению с TCP протокол, не обеспечивающий никаких гарантий доставки и упорядочения данных. Однако, благодаря отсутствию подтверждений он является более быстрым, что позволяет использовать его там, где нужна максимально быстрая доставка.
\subsection{Некоторые приложения}
\begin{itemize}
	\item Голосовые приложения (VoIP);
	\item DNS;
	\item TFTP (обеспечивает надежную доставку своими средствами)
\end{itemize}

\subsection{Литература}
\begin{itemize}
	\item ICND1 \cite{icnd1eng}, Глава 5;
	\item Олифер \cite{olipher}, Глава 17, с. 554-570.
\end{itemize}
% !TeX spellcheck = ru_RU
\section{Физический уровень}
\subsection{Повторители}
Из-за наличия затухания в среде существуют ограничения на длину используемых кабелей. Для построения сильно распределенных сетей можно использовать устройства, называемые повторителями или репитерами. Они усиливают и очищают сигнал, при этом не зная ничего о самом передаваемом фрейме, т.е. работая только с электрическим сигналом. Отсюда следует, что повторители работают на физическом уровне модели OSI.
\begin{figure}[ht!]
	\centering
	\begin{tikzpicture}
		\draw (-4, 0) -- (0,0);
		\draw (0, 0) -- (4,0);
		
		\draw (-3, 0) -- (-3, 1);
		\node at (-3, 1) [label=above:A] {\includegraphics{pic/pc.eps}};
		
		\draw (-2, -1) -- (-2, 0);
		\node at (-2, -1) [label=below:B] {\includegraphics{pic/pc.eps}};
		
		\draw (3, 0) -- (3, 1);
		\node at (3, 1) [label=above:D] {\includegraphics{pic/pc.eps}};
		
		\draw (2, -1) -- (2, 0);
		\node at (2, -1) [label=below:C] {\includegraphics{pic/pc.eps}};
		
		\draw[->] (-3.9, 1.5) -- (-3.9, 0.45);
		\draw[->] (-2.9, -1.5) -- (-2.9, -0.45);
		
		\draw[->] (-4, -0.6) -- (-2.5, 0);
		
		\node at (-4, -0.5) [below] {Коллизия};
		
		\node at (0, 0) [label=above:Повторитель] {\includegraphics{pic/repeater.eps}};
	\end{tikzpicture}
	\caption{Пример применения повторителя}
	\label{fig:repeater}
\end{figure}

Обратимся к рис.~\ref{fig:repeater}. Пусть узлы A и B решили одновременно начать передачу данных и попали в коллизию. Установка повторителя никак на нее влияет, и он просто передаст коллизию во всю сеть. 
\subsection{Концентраторы}
С течением времени на место репитеров пришли концентраторы (хабы). Логика их работы аналогична повторителям, однако полученный сигнал распространяется не через один порт, а через все порты хаба (кроме того, через который сигнал был получен). Использование хабов позволяет значительно увеличить размеры сетевых сегментов. 
\begin{figure}[h!]
	\centering
	\begin{tikzpicture}
		\draw (-4, 0) -- (0,0);
		\draw (0, 0) -- (4,0);
		\draw (0, 0) -- (0, -3);
		
		\draw (-3, 0) -- (-3, 1);
		\node at (-3, 1) [label=above:A] {\includegraphics{pic/pc.eps}};
		
		\draw (-2.25, -1) -- (-2.25, 0);
		\node at (-2.25, -1) [label=below:B] {\includegraphics{pic/pc.eps}};
		
		\draw (-1.5, 0) -- (-1.5, 1);
		\node at (-1.5, 1) [label=above:C] {\includegraphics{pic/pc.eps}};
		
		\draw (3, 0) -- (3, 1);
		\node at (3, 1) [label=above:E] {\includegraphics{pic/pc.eps}};
		
		\draw (2, -1) -- (2, 0);
		\node at (2, -1) [label=below:D] {\includegraphics{pic/pc.eps}};
		
		\draw (0, -2) -- (-1, -2);
		\node at (-1, -2) [label=below:F] {\includegraphics{pic/pc.eps}};
		
		\node at (0, 0) [label=above:Хаб] {\includegraphics{pic/hub.eps}};
	\end{tikzpicture}
	\caption{Пример применения хаба}
	\label{fig:hub}
\end{figure}

Однако концентраторы обладают рядом существенных недостатков. Обратимся к рис.~\ref{fig:hub}. Фрейм, отправленный от A к С, получат все устройства, но, так как MAC-адрес получателя относится только к C, то все остальные должны просто отбросить полученный кадр (но это только в теории!), то есть фрейм будет получен всеми сетевыми устройствами независимо от того, кому он был предназначен. Также, аналогично повторителям, хабы никак не защищают от коллизий.
\subsection{Симплекс, полудуплекс и дуплекс}
Существует три способа передачи данных между двумя устройствами:
\begin{itemize}
	\item симплексный --- передача данных возможна лишь в одну сторону;
	\item полудуплексный --- одновременная передача данных возможна лишь в одном направлении;
	\item полнодуплексный (англ. full-duplex) --- одновременная передача данных возможна в обоих направлениях.
\end{itemize}
Сети, построенные на основе топологии \enquote{Шина}, всегда используют полудуплексную передачу. Стоит отметить, что при использовании полнодуплексного способа связи коллизии невозможны в принципе.
\subsection{Литература}
\begin{itemize}
	\item ICND1 \cite{icnd1eng}, Глава 2.
\end{itemize}